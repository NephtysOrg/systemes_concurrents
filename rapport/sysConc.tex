\documentclass[11pt,a4paper]{article}
\usepackage[top=2cm,bottom=2.5cm,left=2.5cm,right=2.5cm]{geometry}
\usepackage[french]{babel}
\usepackage{graphicx}
\usepackage[utf8]{inputenc}
\usepackage{verbatim}
\usepackage{fancyhdr}
\usepackage[pdfborder={0 0 0}]{hyperref}
\usepackage{listings}
\lstset{
    basicstyle=\fontsize{8}{10}\selectfont\ttfamily
}
\fancyhead[L]{}
\fancyhead[R]{}
\renewcommand{\footrulewidth}{0.5pt}
\fancyfoot[L]{\textit{Système concurrents}}
\fancyfoot[R]{}
\title{\textbf{Rapport de TP : Systèmes concurrents}}
\author{Charles Follet, Roland Bary}
\date{Projet disponible sur  \url{https://github.com/cfollet/ada_concurr}}
\begin{document}
\maketitle
\begin{center}
    \maketitle{\textbf{Université de Pau et des Pays de l'Adour}}
    \begin{center}
        \includegraphics[scale=0.3]{logoUppa.png}
    \end{center}
\end{center}
\thispagestyle{empty}
\newpage
\pagestyle{fancy}
\renewcommand{\contentsname}{Sommaire}
\tableofcontents
\newpage
\section*{Introduction}
\part{Rendez-vous ADA}
L'objectif ici est de résoudre à l'aide de l'outil RDV ADA les problèmes de:
\begin{itemize}
    \item Producteur/Consommateur
          \begin{itemize}
            \item[•]1 Producteur/ 1Consommateur: tampon de taille N
            \item[•]1 producteur/1 Consommateur avec un tampon de taille 1
          \end{itemize}
    \item Lecteurs/Rédacteurs
          \begin{itemize}
            \item[•] Priorité aux Lecteurs
            \item[•] Priorité aux Rédacteurs
            \item[•] Priorité égales
          \end{itemize}
\end{itemize}
\section{Producteur/ Consommateur}
\subsection{1 Producteur/ 1 Consommateur: tampon de taille N}
\begin{scriptsize}
    \verbatiminput{../prod_cons/rdv_ada/prodcons.adb}
\end{scriptsize}
\newpage
\subsection{Producteur/ Consommateur avec un tampon de taille 1}
\begin{scriptsize}
    \verbatiminput{../prod_cons/rdv_ada/prodcons_un.adb}
\end{scriptsize}
\newpage
\section{Lecteurs/ Rédacteurs}
\subsection{Priorité aux Lecteurs}
\begin{scriptsize}
    \verbatiminput{../lect_red/rdv_ada/prio_lecteur/lecteurredacteur.adb}
\end{scriptsize}
\newpage
\subsection{Priorité aux Rédacteurs}
\begin{scriptsize}
    \verbatiminput{../lect_red/rdv_ada/prio_redacteur/lecteurredacteur.adb}
\end{scriptsize}
\newpage
\subsection{Priorités égales}
\subsubsection{Principe de la solution}
Une solution naïve au problème aurait été de directement utiliser la solution avec les compteurs de Robert vu en cours, c'est à dire : 
\begin{description}
    \item[Lecture] \#act(ECRIRE) = 0
    \item[Ecriture] \#act(ECRIRE) + \#act(LIRE) = 0
\end{description}
Si on utilise ces conditions, le cas suivant ne fonctionne pas :\\~\\
L1 est en cours\\
\indent R1 arrive \\
\indent L2 arrive\\~\\
En effet, rien n’empêche L2 de passer en lecture au détriment de R1.\\
Pour résoudre ce problème il faut que R1 signale son intention d'entrer en SC et nous avons choisis d'utiliser un drapeau.\\
Un lecteur/redacteur désirant aller en SC doit lever le drapeau(drapeau:= 1). Ainsi, un lecteur/redacteur voulant entrer doit disposer de ce drapeau et le problème précédent est résolu.
\subsubsection{Code ADA}
\begin{scriptsize}
    \verbatiminput{../lect_red/rdv_ada/prio_egale/lecteurredacteur.adb}
\end{scriptsize}
\subsubsection{Tests}
\underline{Test 1 :} \\
L1 est en cours\\
\indent R1 arrive \\
\indent L2 arrive\\
\begin{center}
    
    \begin{tabular}{|c|c|c|c|}
        \hline 
        \rule[-1ex]{0pt}{2.5ex} \textbf{Actions} & \textbf{nbLect}       & \textbf{nbRed} & \textbf{drapeau} \\ 
        \hline 
        \rule[-1ex]{0pt}{2.5ex} L1.LIRE          & 1                     & 0              & 0                \\ 
        \hline 
        \rule[-1ex]{0pt}{2.5ex} arrivée de R1   &                       &                &                  \\ 
        \hline 
        \rule[-1ex]{0pt}{2.5ex} R1.barriere      & 1                     & 0              & 1                \\ 
        \hline 
        \rule[-1ex]{0pt}{2.5ex} R1.debutRed      & R1 bloqué a debutRed &                &                  \\ 
        \hline 
        \rule[-1ex]{0pt}{2.5ex} arrivée de L2   &                       &                &                  \\ 
        \hline 
        \rule[-1ex]{0pt}{2.5ex} L2.barriere      & L2 bloqué a barriere &                &                  \\ 
        \hline 
        \rule[-1ex]{0pt}{2.5ex} L1.finLect       & 0                     & 0              & 1                \\ 
        \hline 
        \rule[-1ex]{0pt}{2.5ex} R1.debutRed      & 0                     & 1              & 1                \\ 
        \hline 
        \rule[-1ex]{0pt}{2.5ex} R1.finRed        & 0                     & 0              & 0                \\ 
        \hline 
        \rule[-1ex]{0pt}{2.5ex} L2.debutLect     & 1                     & 0              & 0                \\ 
        \hline 
        \rule[-1ex]{0pt}{2.5ex} L2.finLect       & 0                     & 0              & 0                \\ 
        \hline 
    \end{tabular} 
\end{center}
\newpage

\underline{Test 2 :} \\
R1 est en cours\\
\indent L1 arrive \\
\indent R2 arrive\\
\begin{center}
    \begin{tabular}{|c|c|c|c|}
        \hline 
        \rule[-1ex]{0pt}{2.5ex} \textbf{Actions} & \textbf{nbLect}              & \textbf{nbRed} & \textbf{drapeau} \\ 
        \hline 
        \rule[-1ex]{0pt}{2.5ex} R1.ecrire        & 0                            & 1              & 1                \\ 
        \hline 
        \rule[-1ex]{0pt}{2.5ex} arrivée de L1   &                              &                &                  \\ 
        \hline 
        \rule[-1ex]{0pt}{2.5ex} L1.barriere      & L1 bloqué a barriere (FIFO) &                &                  \\ 
        \hline 
        \rule[-1ex]{0pt}{2.5ex} arrivée de R2   &                              &                &                  \\ 
        \hline 
        \rule[-1ex]{0pt}{2.5ex} R2.barriere      & R2 bloqué a barriere (FIFO) &                &                  \\ 
        \hline 
        \rule[-1ex]{0pt}{2.5ex} R1.finRed        & 0                            & 0              & 0                \\ 
        \hline 
        \rule[-1ex]{0pt}{2.5ex} L1.debutLect     & 1                            & 0              & 0                \\ 
        \hline 
        \rule[-1ex]{0pt}{2.5ex} R2.debutRed      & R2 bloqué a debutRed        &                &                  \\ 
        \hline 
        \rule[-1ex]{0pt}{2.5ex} L1.finLect       & 0                            & 0              & 0                \\ 
        \hline 
        \rule[-1ex]{0pt}{2.5ex} R2.debutRed      & 0                            & 1              & 1                \\ 
        \hline 
        \rule[-1ex]{0pt}{2.5ex} R2.finRed        & 0                            & 0              & 0                \\ 
        \hline 
    \end{tabular} 
\end{center}
\newpage




\part{Objets/Types protégés ADA 95}
L'objectif de cette partie est de :
\begin{itemize}
    \item Rappeler le principe des Objets/Types protégés en ADA 95 : définition, principe, sémantique, différence avec les packages ADA
    \item Comparer les  Objets/types protégés a d'autres outils comme  les sémaphores, RDV ADA, : pouvoir d'expression, difficulté/facilité d'utilisation, etc..
    \item  Implémenter à l'aide Objets/Types protégés:
          \begin{itemize}
            \item[•] Producteur/Consommateur
            \item[•] Lecteurs/ Rédacteurs: sans priorité et  priorité aux lecteurs
            \item[•] Le problème du Carrefour a sens giratoire  
          \end{itemize}
\end{itemize}
%%Beaucoup plus simples que les moniteurs de hoare
\section{Rappel des principes des Objets/Types protégés en ADA 95}
\subsection{Définition}
\subsection{Principe}
\subsection{Sémantique}
\subsection{Différence avec les packages ADA}
\section{Implémentation à l'aide des Objets/Types protégés}
\newpage
\subsection{Producteur/ Consommateur}
\begin{scriptsize}
    \verbatiminput{../prod_cons/objet_protege/prodcons.adb}
\end{scriptsize}
\newpage
\subsection{Lecteurs/ Rédacteurs: sans priorité et priorité lecteurs en ADA 95}
\subsubsection{Priorité égale}
\begin{scriptsize}
    \verbatiminput{../lect_red/objet_protege/prio_egale/lecteurredacteur.adb}
\end{scriptsize}
\newpage
\subsubsection{Priorité lecteurs}
\begin{scriptsize}
    \verbatiminput{../lect_red/objet_protege/prio_lecteur/lecteurredacteur.adb}
\end{scriptsize}
\newpage
\subsection{Le problème du carrefour à sens giratoire}
\begin{scriptsize}
    \verbatiminput{../careffour/giratoire.adb}
\end{scriptsize}
\end{document}